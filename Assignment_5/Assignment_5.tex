\documentclass[journal,12pt,twocolumn]{IEEEtran}

\usepackage{setspace}
\usepackage{gensymb}

\singlespacing


\usepackage[cmex10]{amsmath}

\usepackage{amsthm}

\usepackage{mathrsfs}
\usepackage{txfonts}
\usepackage{stfloats}
\usepackage{bm}
\usepackage{cite}
\usepackage{cases}
\usepackage{subfig}

\usepackage{longtable}
\usepackage{multirow}

\usepackage{enumitem}
\usepackage{mathtools}
\usepackage{steinmetz}
\usepackage{tikz}
\usepackage{circuitikz}
\usepackage{verbatim}
\usepackage{tfrupee}
\usepackage[breaklinks=true]{hyperref}

\usepackage{tkz-euclide}

\usetikzlibrary{calc,math}
\usepackage{listings}
    \usepackage{color}                                            %%
    \usepackage{array}                                            %%
    \usepackage{longtable}                                        %%
    \usepackage{calc}                                             %%
    \usepackage{multirow}                                         %%
    \usepackage{hhline}                                           %%
    \usepackage{ifthen}                                           %%
    \usepackage{lscape}     
\usepackage{multicol}
\usepackage{chngcntr}

\DeclareMathOperator*{\Res}{Res}

\renewcommand\thesection{\arabic{section}}
\renewcommand\thesubsection{\thesection.\arabic{subsection}}
\renewcommand\thesubsubsection{\thesubsection.\arabic{subsubsection}}

\renewcommand\thesectiondis{\arabic{section}}
\renewcommand\thesubsectiondis{\thesectiondis.\arabic{subsection}}
\renewcommand\thesubsubsectiondis{\thesubsectiondis.\arabic{subsubsection}}


\hyphenation{op-tical net-works semi-conduc-tor}
\def\inputGnumericTable{}                                 %%

\lstset{
%language=C,
frame=single, 
breaklines=true,
columns=fullflexible
}
\begin{document}


\newtheorem{theorem}{Theorem}[section]
\newtheorem{problem}{Problem}
\newtheorem{proposition}{Proposition}[section]
\newtheorem{lemma}{Lemma}[section]
\newtheorem{corollary}[theorem]{Corollary}
\newtheorem{example}{Example}[section]
\newtheorem{definition}[problem]{Definition}

\newcommand{\BEQA}{\begin{eqnarray}}
\newcommand{\EEQA}{\end{eqnarray}}
\newcommand{\define}{\stackrel{\triangle}{=}}
\bibliographystyle{IEEEtran}
\providecommand{\mbf}{\mathbf}
\providecommand{\pr}[1]{\ensuremath{\Pr\left(#1\right)}}
\providecommand{\qfunc}[1]{\ensuremath{Q\left(#1\right)}}
\providecommand{\sbrak}[1]{\ensuremath{{}\left[#1\right]}}
\providecommand{\lsbrak}[1]{\ensuremath{{}\left[#1\right.}}
\providecommand{\rsbrak}[1]{\ensuremath{{}\left.#1\right]}}
\providecommand{\brak}[1]{\ensuremath{\left(#1\right)}}
\providecommand{\lbrak}[1]{\ensuremath{\left(#1\right.}}
\providecommand{\rbrak}[1]{\ensuremath{\left.#1\right)}}
\providecommand{\cbrak}[1]{\ensuremath{\left\{#1\right\}}}
\providecommand{\lcbrak}[1]{\ensuremath{\left\{#1\right.}}
\providecommand{\rcbrak}[1]{\ensuremath{\left.#1\right\}}}
\theoremstyle{remark}
\newtheorem{rem}{Remark}
\newcommand{\sgn}{\mathop{\mathrm{sgn}}}
\providecommand{\abs}[1]{\left\vert#1\right\vert}
\providecommand{\res}[1]{\Res\displaylimits_{#1}} 
\providecommand{\norm}[1]{\left\lVert#1\right\rVert}
%\providecommand{\norm}[1]{\lVert#1\rVert}
\providecommand{\mtx}[1]{\mathbf{#1}}
\providecommand{\mean}[1]{E\left[ #1 \right]}
\providecommand{\fourier}{\overset{\mathcal{F}}{ \rightleftharpoons}}
%\providecommand{\hilbert}{\overset{\mathcal{H}}{ \rightleftharpoons}}
\providecommand{\system}{\overset{\mathcal{H}}{ \longleftrightarrow}}
	%\newcommand{\solution}[2]{\textbf{Solution:}{#1}}
\newcommand{\solution}{\noindent \textbf{Solution: }}
\newcommand{\cosec}{\,\text{cosec}\,}
\providecommand{\dec}[2]{\ensuremath{\overset{#1}{\underset{#2}{\gtrless}}}}
\newcommand{\myvec}[1]{\ensuremath{\begin{pmatrix}#1\end{pmatrix}}}
\newcommand{\mydet}[1]{\ensuremath{\begin{vmatrix}#1\end{vmatrix}}}
\numberwithin{equation}{subsection}
\makeatletter
\@addtoreset{figure}{problem}
\makeatother
\let\StandardTheFigure\thefigure
\let\vec\mathbf
\renewcommand{\thefigure}{\theproblem}
\def\putbox#1#2#3{\makebox[0in][l]{\makebox[#1][l]{}\raisebox{\baselineskip}[0in][0in]{\raisebox{#2}[0in][0in]{#3}}}}
     \def\rightbox#1{\makebox[0in][r]{#1}}
     \def\centbox#1{\makebox[0in]{#1}}
     \def\topbox#1{\raisebox{-\baselineskip}[0in][0in]{#1}}
     \def\midbox#1{\raisebox{-0.5\baselineskip}[0in][0in]{#1}}
\vspace{3cm}
\title{EE5609 Assignment 5}
\author{Abhishek Thakur}
\maketitle
\newpage
\bigskip
\renewcommand{\thefigure}{\theenumi}
\renewcommand{\thetable}{\theenumi}
\begin{abstract}
This document contains the solution of geometry through linear algebra.
\end{abstract}
Download latex and python codes from 
\begin{lstlisting}
https://github.com/abhishekt711/EE5609/tree/master/Assignment_5
\end{lstlisting}
%
\section{Problem}
Prove that the following equations represent two straight lines, find also their point of intersection and the angle between them.\\
$6y^2-xy-x^2+30y+36=0$.
\section{Explanation}
The given equation can be written as:
\begin{align}
-x^2-xy+6y^2+30y+36=0 \label{eq1}
\end{align}
\mydet{\vec{V}&\vec{u}\\\vec{u}^T&f} of \eqref{eq1} becomes
\begin{align}
    \mydet{-1&-\frac{1}{2}&0\\\frac{-1}{2}&6&15\\0&15&36}=0\label{eq02}
\end{align}
Expanding equation \eqref{eq02}, we get zero.\\
Hence given equation represents a pair of straight lines.

The general equation second degree is given by
\begin{equation}\label{eq5}
	ax^2 + 2bxy + cy^2 + 2dx + 2ey + f = 0
\end{equation}
Let $(\alpha,\beta)$ be their point of intersection, then
\begin{equation}\label{eq6}
	\myvec{ a & h\\ h & b}\myvec{\alpha \\ \beta} = \myvec{-d \\ -e}
\end{equation}
Given equation is
\begin{align}
	-x^2-xy+6y^2+30y+36=0
\end{align}
Substituting in \eqref{eq6}
\begin{align}
	\label{eq16}\myvec{ -1 & \frac{-1}{2}\\\frac{-1}{2} & 6}\myvec{\alpha \\ \beta} = \myvec{0 \\ -15} \\
	\label{eq17}\implies \myvec{\alpha \\ \beta} = \myvec{\frac{30}{23} \\ \frac{-60}{23}}
\end{align}
From, Spectral decomposition,
\begin{align}
	\vec{V} &= \vec{P}\vec{D}\vec{P}^T\\
	\label{eq18}\vec{V} &= \myvec{ -1 & \frac{-1}{2}\\\frac{-1}{2} & 6}\\
	\label{eq19}\vec{P} &= \myvec{7-5 \sqrt{2} & 7+5 \sqrt{2}\\ 1 & 1}\\
	\label{eq20}\vec{D} &= \myvec{\frac{5+5\sqrt{2}}{2} & 0\\ 0 & \frac{5-5\sqrt{2}}{2}}
\end{align}
P and D are also verified using python code from
\begin{lstlisting}
codes/diagonalize1.py
\end{lstlisting}
Using, \eqref{eq17}, \eqref{eq19} and \eqref{eq20} in,
\begin{align}
	u_1(x-\alpha) + u_2(y-\beta) &= \pm \sqrt{-\frac{\lambda_2}{\lambda_1}}(v_1(x-\alpha) + v_2(y-\beta))\label{eq14}
\end{align}
\begin{multline}\label{eq21}
\implies	\left(7-5 \sqrt{2}\right)\left(x-\frac{30}{23}\right) + \left(y+\frac{60}{23}\right) \\= \pm \sqrt{-\frac{\frac{5-5\sqrt{2}}{2}}{\frac{5+5\sqrt{2}}{2}}}\left(\left(7-5 \sqrt{2}\right)\left(x-\frac{30}{23}\right) + \left(y+\frac{60}{23}\right)\right)
\end{multline}
simplifying \ref{eq21}, we get:
\begin{align}
	\label{eq22}-x + 2y + 6 = 0 \text{ and } x + 3y + 6 = 0\\
	\implies (-x + 2y + 6)(x + 3y + 6) = 0\\
	\therefore -x+2y=-6 \quad , \quad x+3y=-6\label{eq23}
\end{align}
The augmented matrix for the set of equations represented in \eqref{eq23} is
\begin{align}
\myvec{-1&2&-6\\1&3&-6}
\end{align}
Row reducing the matrix
\begin{align}
 \myvec{-1&2&-6\\1&3&-6}\xleftrightarrow[]{R_2\leftarrow R_2+R_1}\myvec{-1&2&-6\\0&5&-12}\\
 \xleftrightarrow[]{R_1\leftarrow R_1-\frac{2}{5}\times R_2}\myvec{-1&0&-\frac{6}{5}\\0&5&-12}\\
 \xleftrightarrow[]{R_1\leftarrow -1\times R_1  , R_2\leftarrow \frac{1}{5}\times R_2}\myvec{1&0&\frac{6}{5}\\0&1&-\frac{12}{5}}
\end{align}
\text{Hence, the intersection point is}
\myvec{\frac{6}{5}\\-\frac{12}{5}}\\
Also, Verified using python code from
\begin{lstlisting}
codes/Assignment_5.py
\end{lstlisting}
Angle between two lines, $\theta$ can be given by
\begin{align}
n_1&=\myvec{-2\\-1}\\
n_2&=\myvec{-3\\1}\\
\cos \theta &= \frac{\vec{n_1}^T\vec{n_2}}{\norm{\vec{n_1}}\norm{\vec{n_2}}}\\
\cos \theta&=\frac{\myvec{-2&-1}\myvec{-3\\1}}{\sqrt{(-2)^2 +(-1)^2} \times \sqrt{+(-3)^2+1}}=\frac{1}{\sqrt{2}}\\
\implies \theta &= 45\degree
\end{align}
\begin{figure}[!h]
\centering
\includegraphics[width=\columnwidth]{Figure_1.png}
\caption{plot showing intersection of lines.}
\label{Fig_1}
\end{figure}
\end{document}