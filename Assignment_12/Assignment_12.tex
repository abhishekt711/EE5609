\documentclass[journal,12pt,twocolumn]{IEEEtran}

\usepackage{setspace}
\usepackage{gensymb}

\singlespacing


\usepackage[cmex10]{amsmath}

\usepackage{amsthm}

\usepackage{mathrsfs}
\usepackage{txfonts}
\usepackage{stfloats}
\usepackage{bm}
\usepackage{cite}
\usepackage{cases}
\usepackage{subfig}

\usepackage{longtable}
\usepackage{multirow}

\usepackage{enumitem}
\usepackage{mathtools}
\usepackage{steinmetz}
\usepackage{tikz}
\usepackage{circuitikz}
\usepackage{verbatim}
\usepackage{tfrupee}
\usepackage[breaklinks=true]{hyperref}

\usepackage{tkz-euclide}

\usetikzlibrary{calc,math}
\usepackage{listings}
    \usepackage{color}                                            %%
    \usepackage{array}                                            %%
    \usepackage{longtable}                                        %%
    \usepackage{calc}                                             %%
    \usepackage{multirow}                                         %%
    \usepackage{hhline}                                           %%
    \usepackage{ifthen}                                           %%
    \usepackage{lscape}     
\usepackage{multicol}
\usepackage{chngcntr}

\DeclareMathOperator*{\Res}{Res}

\renewcommand\thesection{\arabic{section}}
\renewcommand\thesubsection{\thesection.\arabic{subsection}}
\renewcommand\thesubsubsection{\thesubsection.\arabic{subsubsection}}

\renewcommand\thesectiondis{\arabic{section}}
\renewcommand\thesubsectiondis{\thesectiondis.\arabic{subsection}}
\renewcommand\thesubsubsectiondis{\thesubsectiondis.\arabic{subsubsection}}


\hyphenation{op-tical net-works semi-conduc-tor}
\def\inputGnumericTable{}                                 %%

\lstset{
%language=C,
frame=single, 
breaklines=true,
columns=fullflexible
}
\begin{document}


\newtheorem{theorem}{Theorem}[section]
\newtheorem{problem}{Problem}
\newtheorem{proposition}{Proposition}[section]
\newtheorem{lemma}{Lemma}[section]
\newtheorem{corollary}[theorem]{Corollary}
\newtheorem{example}{Example}[section]
\newtheorem{definition}[problem]{Definition}

\newcommand{\BEQA}{\begin{eqnarray}}
\newcommand{\EEQA}{\end{eqnarray}}
\newcommand{\define}{\stackrel{\triangle}{=}}
\bibliographystyle{IEEEtran}
\providecommand{\mbf}{\mathbf}
\providecommand{\pr}[1]{\ensuremath{\Pr\left(#1\right)}}
\providecommand{\qfunc}[1]{\ensuremath{Q\left(#1\right)}}
\providecommand{\sbrak}[1]{\ensuremath{{}\left[#1\right]}}
\providecommand{\lsbrak}[1]{\ensuremath{{}\left[#1\right.}}
\providecommand{\rsbrak}[1]{\ensuremath{{}\left.#1\right]}}
\providecommand{\brak}[1]{\ensuremath{\left(#1\right)}}
\providecommand{\lbrak}[1]{\ensuremath{\left(#1\right.}}
\providecommand{\rbrak}[1]{\ensuremath{\left.#1\right)}}
\providecommand{\cbrak}[1]{\ensuremath{\left\{#1\right\}}}
\providecommand{\lcbrak}[1]{\ensuremath{\left\{#1\right.}}
\providecommand{\rcbrak}[1]{\ensuremath{\left.#1\right\}}}
\theoremstyle{remark}
\newtheorem{rem}{Remark}
\newcommand{\sgn}{\mathop{\mathrm{sgn}}}
\providecommand{\abs}[1]{\left\vert#1\right\vert}
\providecommand{\res}[1]{\Res\displaylimits_{#1}} 
\providecommand{\norm}[1]{\left\lVert#1\right\rVert}
%\providecommand{\norm}[1]{\lVert#1\rVert}
\providecommand{\mtx}[1]{\mathbf{#1}}
\providecommand{\mean}[1]{E\left[ #1 \right]}
\providecommand{\fourier}{\overset{\mathcal{F}}{ \rightleftharpoons}}
%\providecommand{\hilbert}{\overset{\mathcal{H}}{ \rightleftharpoons}}
\providecommand{\system}{\overset{\mathcal{H}}{ \longleftrightarrow}}
	%\newcommand{\solution}[2]{\textbf{Solution:}{#1}}
\newcommand{\solution}{\noindent \textbf{Solution: }}
\newcommand{\cosec}{\,\text{cosec}\,}
\providecommand{\dec}[2]{\ensuremath{\overset{#1}{\underset{#2}{\gtrless}}}}
\newcommand{\myvec}[1]{\ensuremath{\begin{pmatrix}#1\end{pmatrix}}}
\newcommand{\mydet}[1]{\ensuremath{\begin{vmatrix}#1\end{vmatrix}}}
\numberwithin{equation}{subsection}
\makeatletter
\@addtoreset{figure}{problem}
\makeatother
\let\StandardTheFigure\thefigure
\let\vec\mathbf
\renewcommand{\thefigure}{\theproblem}
\def\putbox#1#2#3{\makebox[0in][l]{\makebox[#1][l]{}\raisebox{\baselineskip}[0in][0in]{\raisebox{#2}[0in][0in]{#3}}}}
     \def\rightbox#1{\makebox[0in][r]{#1}}
     \def\centbox#1{\makebox[0in]{#1}}
     \def\topbox#1{\raisebox{-\baselineskip}[0in][0in]{#1}}
     \def\midbox#1{\raisebox{-0.5\baselineskip}[0in][0in]{#1}}
\vspace{3cm}
\title{EE5609 Assignment 12}
\author{Abhishek Thakur}
\maketitle
\newpage
\bigskip
\renewcommand{\thefigure}{\theenumi}
\renewcommand{\thetable}{\theenumi}
\begin{abstract}
This document solves problem based on solution of vector space.
\end{abstract}
Download all solutions from 
\begin{lstlisting}
https://github.com/abhishekt711/EE5609/tree/master/Assignment_12
\end{lstlisting}
\section{Problem}
In $R^3$, let $\alpha_1 = \myvec{1\\0\\1}$, $\alpha_2 =\myvec{0\\1\\-2}$ and 
$\alpha_3 =\myvec{-1\\-1\\0}$. 
if f is a linear functional on $R^3$ such that\\ $f(\alpha_1)=1$, $f(\alpha_2)=-1$,  $f(\alpha_3)=3$,\\ And if $\alpha$=\myvec{a\\b\\c}, find $f(\alpha)$.
\section{Solution}
Given, $\alpha = \myvec{a\\b\\c}$.
Let,
\begin{align}
\vec{A}=\myvec{\alpha_1&\alpha_2&\alpha_3}\\
\vec{A}\vec{X}=\alpha\\
\myvec{1 & 0 & -1\\0 & 1 & -1\\1 & -2 & 0}\myvec{x_1\\x_2\\x_3} = \myvec{a\\b\\c}
\end{align}
$ \vec{X}=A^{-1}\alpha$ will give solution of the equation.
\begin{align}
\myvec{1 & 0 & -1 & 1 & 0 & 0\\
0 & 1 & -1 & 0 & 1 & 0\\
1 & -2 & 0 & 0 & 0 & 1}\xleftrightarrow[]{R_3\leftarrow R_3-R_1}\\
\myvec{1 & 0 & -1 & 1 & 0 & 0\\
0 & 1 & -1 & 0 & 1 & 0\\
0 & -2 & 1 & -1 & 0 & 1}
\xleftrightarrow[]{R_3\leftarrow R_3+2R_2}\\
\myvec{1 & 0 & -1 & 1 & 0 & 0\\
0 & 1 & -1 & 0 & 1 & 0\\
0 & 0 & -1 & -1 & 2 & 1}\xleftrightarrow[]{R_3\leftarrow R_3/(-1)}\\
\myvec{1 & 0 & -1 & 1 & 0 & 0\\
0 & 1 & -1 & 0 & 1 & 0\\
0 & 0 & 1 & 1 & -2 & -1}\xleftrightarrow{R_1\leftarrow R_1+R_3}\\
\myvec{1 & 0 & 0 & 2 & -2 & -1\\
0 & 1 & -1 & 0 & 1 & 0\\
0 & 0 & 1 & 1 & -2 & -1}\xleftrightarrow{R_2\leftarrow R_2+R_3}\\
\myvec{1 & 0 & 0 & 2 & -2 & -1\\
0 & 1 & 0 & 1 & -1 & -1\\
0 & 0 & 1 & 1 & -2 & -1}
\end{align}
Thus,
\begin{align}
    A^{-1}=\myvec{
    2 &-2&-1\\
    1&-1&-1\\
    1&-2&-1
    }\\
\vec{X} = A^{-1}\alpha=\myvec{2 &-2&-1\\1&-1&-1\\1&-2&-1}
\myvec{a\\b\\c}
\end{align}
Given, $f$ is a linear functional on $R^3$,
\begin{align}
\vec{\alpha} = \alpha_1 x_1 + \alpha_2 x_2 + \alpha_3 x_3\\
\implies f(\alpha) = \vec{X}^T \myvec{f(\alpha_1)\\f(\alpha_2)\\f(\alpha_3)}
\end{align}
Given, $f(\alpha_1)=1$, $f(\alpha_2)=-1$ and $f(\alpha_3)=3$.
\begin{align}
 f(\alpha) &= \vec{X}^T \myvec{1\\-1\\3}\\
\implies f(\alpha) &= \myvec{a\\b\\c}^T\myvec{2 &1&1\\-2&-1&-2 \\-1&-1&-1}\myvec{1\\-1\\3}\\
f(\alpha)&=\myvec{2a-2b-c \\ a-b-c \\ a-2b-c}^T\myvec{1\\-1\\3}
\end{align}
Hence,
\begin{align}
f(\alpha)=4a-7b-3c
\end{align}
\end{document}