\documentclass[journal,12pt,twocolumn]{IEEEtran}

\usepackage{setspace}
\usepackage{gensymb}

\singlespacing


\usepackage[cmex10]{amsmath}

\usepackage{amsthm}

\usepackage{mathrsfs}
\usepackage{txfonts}
\usepackage{stfloats}
\usepackage{bm}
\usepackage{cite}
\usepackage{cases}
\usepackage{subfig}

\usepackage{longtable}
\usepackage{multirow}

\usepackage{enumitem}
\usepackage{mathtools}
\usepackage{steinmetz}
\usepackage{tikz}
\usepackage{circuitikz}
\usepackage{verbatim}
\usepackage{tfrupee}
\usepackage[breaklinks=true]{hyperref}

\usepackage{tkz-euclide}

\usetikzlibrary{calc,math}
\usepackage{listings}
    \usepackage{color}                                            %%
    \usepackage{array}                                            %%
    \usepackage{longtable}                                        %%
    \usepackage{calc}                                             %%
    \usepackage{multirow}                                         %%
    \usepackage{hhline}                                           %%
    \usepackage{ifthen}                                           %%
    \usepackage{lscape}     
\usepackage{multicol}
\usepackage{chngcntr}

\DeclareMathOperator*{\Res}{Res}

\renewcommand\thesection{\arabic{section}}
\renewcommand\thesubsection{\thesection.\arabic{subsection}}
\renewcommand\thesubsubsection{\thesubsection.\arabic{subsubsection}}

\renewcommand\thesectiondis{\arabic{section}}
\renewcommand\thesubsectiondis{\thesectiondis.\arabic{subsection}}
\renewcommand\thesubsubsectiondis{\thesubsectiondis.\arabic{subsubsection}}


\hyphenation{op-tical net-works semi-conduc-tor}
\def\inputGnumericTable{}                                 %%

\lstset{
%language=C,
frame=single, 
breaklines=true,
columns=fullflexible
}
\begin{document}


\newtheorem{theorem}{Theorem}[section]
\newtheorem{problem}{Problem}
\newtheorem{proposition}{Proposition}[section]
\newtheorem{lemma}{Lemma}[section]
\newtheorem{corollary}[theorem]{Corollary}
\newtheorem{example}{Example}[section]
\newtheorem{definition}[problem]{Definition}

\newcommand{\BEQA}{\begin{eqnarray}}
\newcommand{\EEQA}{\end{eqnarray}}
\newcommand{\define}{\stackrel{\triangle}{=}}
\bibliographystyle{IEEEtran}
\providecommand{\mbf}{\mathbf}
\providecommand{\pr}[1]{\ensuremath{\Pr\left(#1\right)}}
\providecommand{\qfunc}[1]{\ensuremath{Q\left(#1\right)}}
\providecommand{\sbrak}[1]{\ensuremath{{}\left[#1\right]}}
\providecommand{\lsbrak}[1]{\ensuremath{{}\left[#1\right.}}
\providecommand{\rsbrak}[1]{\ensuremath{{}\left.#1\right]}}
\providecommand{\brak}[1]{\ensuremath{\left(#1\right)}}
\providecommand{\lbrak}[1]{\ensuremath{\left(#1\right.}}
\providecommand{\rbrak}[1]{\ensuremath{\left.#1\right)}}
\providecommand{\cbrak}[1]{\ensuremath{\left\{#1\right\}}}
\providecommand{\lcbrak}[1]{\ensuremath{\left\{#1\right.}}
\providecommand{\rcbrak}[1]{\ensuremath{\left.#1\right\}}}
\theoremstyle{remark}
\newtheorem{rem}{Remark}
\newcommand{\sgn}{\mathop{\mathrm{sgn}}}
\providecommand{\abs}[1]{\left\vert#1\right\vert}
\providecommand{\res}[1]{\Res\displaylimits_{#1}} 
\providecommand{\norm}[1]{\left\lVert#1\right\rVert}
%\providecommand{\norm}[1]{\lVert#1\rVert}
\providecommand{\mtx}[1]{\mathbf{#1}}
\providecommand{\mean}[1]{E\left[ #1 \right]}
\providecommand{\fourier}{\overset{\mathcal{F}}{ \rightleftharpoons}}
%\providecommand{\hilbert}{\overset{\mathcal{H}}{ \rightleftharpoons}}
\providecommand{\system}{\overset{\mathcal{H}}{ \longleftrightarrow}}
	%\newcommand{\solution}[2]{\textbf{Solution:}{#1}}
\newcommand{\solution}{\noindent \textbf{Solution: }}
\newcommand{\cosec}{\,\text{cosec}\,}
\providecommand{\dec}[2]{\ensuremath{\overset{#1}{\underset{#2}{\gtrless}}}}
\newcommand{\myvec}[1]{\ensuremath{\begin{pmatrix}#1\end{pmatrix}}}
\newcommand{\mydet}[1]{\ensuremath{\begin{vmatrix}#1\end{vmatrix}}}
\numberwithin{equation}{subsection}
\makeatletter
\@addtoreset{figure}{problem}
\makeatother
\let\StandardTheFigure\thefigure
\let\vec\mathbf
\renewcommand{\thefigure}{\theproblem}
\def\putbox#1#2#3{\makebox[0in][l]{\makebox[#1][l]{}\raisebox{\baselineskip}[0in][0in]{\raisebox{#2}[0in][0in]{#3}}}}
     \def\rightbox#1{\makebox[0in][r]{#1}}
     \def\centbox#1{\makebox[0in]{#1}}
     \def\topbox#1{\raisebox{-\baselineskip}[0in][0in]{#1}}
     \def\midbox#1{\raisebox{-0.5\baselineskip}[0in][0in]{#1}}
\vspace{3cm}
\title{EE5609 Assignment 2}
\author{Abhishek Thakur}
\maketitle
\newpage
\bigskip
\renewcommand{\thefigure}{\theenumi}
\renewcommand{\thetable}{\theenumi}
\begin{abstract}
This document contains the solution of complex number using matrix method.
\end{abstract}
Download latex and python codes from 
\begin{lstlisting}
https://github.com/abhishekt711/EE5609/tree/master/Assignment_2
\end{lstlisting}
%
\section{Problem}
Let $\vec{Z_1}=\myvec{2\\ -1} ,  \vec{Z_2}=\myvec{-2\\ 1}$. Find\\
a)Re $\left(\frac{\vec{z_1z_2}}{\vec{z_1}^*}\right)$\\
b)Im $\left(\frac{1}{\vec{z_1}\vec{z_1}^*}\right)$
\section{Explanation}
Any complex number can be expressed in matrix representation as follows:
\begin{align}  
& \myvec{a1\\ a2} = \myvec{a1 & -a2\\ a2 & a1}\myvec{1\\ 0}\label{eq:eqn_1}
\end{align}
and,
\begin{align}
&\vec{a^{-1}}=\myvec{a1 & -a2\\ a2 & a1}^{-1}\myvec{1\\ 0}\label{eq:eqn_2}\\
&\vec{a^{-1}}=\frac{1}{\norm{\vec{a}^2}}\myvec{a_1\\ -a_2}\label{eq:eqn_3}\\
&\left(\frac{\vec{z_1z_2}}{\vec{z_1}^*}\right)=\myvec{2 & 1\\ -1 & 2}\myvec{-2 & -1\\ 1 & -2}\left[\myvec{2 & -1\\ 1 & 2}\right]^{-1}\myvec{1 \\0}\label{eq:eqn_4}\\
&\left(\frac{\vec{z_1z_2}}{\vec{z_1}^*}\right)=\myvec{2 & 1\\ -1 & 2}\myvec{-2 & -1\\ 1 & -2}\left[\frac{1}{5}\myvec{2 & 1\\ -1 & 2}\right]\myvec{1 \\0}\label{eq:eqn_5}\\
&\left(\frac{\vec{z_1z_2}}{\vec{z_1}^*}\right)=\frac{1}{5}\myvec{-2 & -11\\ 11 & -2}\myvec{1 \\0}\label{eq:eqn_6}\\
&\left(\frac{\vec{z_1z_2}}{\vec{z_1}^*}\right)=\frac{1}{5}\myvec{-2 \\11}\label{eq:eqn_7}
\end{align}
Hence, the real part of $\left(\frac{\vec{z_1z_2}}{\vec{z_1}^*}\right)=-\frac{2}{5}$

\begin{align}
&\left(\frac{1}{\vec{z_1}\vec{z_1}^*}\right)=(\vec{z_1}\vec{z_1}^*)^{-1}\label{eq:eqn_8}\\
&\left(\frac{1}{\vec{z_1}\vec{z_1}^*}\right)=\left[\myvec{2 & 1\\ -1 & 2}\myvec{2 & -1\\ 1 & 2}\right]^{-1}\myvec{1 \\0}\label{eq:eqn_9}\\
&\left(\frac{1}{\vec{z_1}\vec{z_1}^*}\right)=\left[\myvec{5 & 0\\ 0 & 5}\right]^{-1}\myvec{1 \\0}\label{eq:eqn_10}\\
&\left(\frac{1}{\vec{z_1}\vec{z_1}^*}\right)=\frac{1}{25}\myvec{5 & 0\\ 0 & 5}\myvec{1 \\0}\label{eq:eqn_11}\\
&\left(\frac{1}{\vec{z_1}\vec{z_1}^*}\right)=\frac{1}{25}\myvec{5 \\ 0}\label{eq:eqn_12}
\end{align}
Hence, the imaginary part of $\left(\frac{1}{\vec{z_1}\vec{z_1}^*}\right)=0$.
\end{document}