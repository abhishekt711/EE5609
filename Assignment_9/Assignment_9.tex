\documentclass[journal,12pt,twocolumn]{IEEEtran}

\usepackage{setspace}
\usepackage{gensymb}

\singlespacing


\usepackage[cmex10]{amsmath}

\usepackage{amsthm}

\usepackage{mathrsfs}
\usepackage{txfonts}
\usepackage{stfloats}
\usepackage{bm}
\usepackage{cite}
\usepackage{cases}
\usepackage{subfig}

\usepackage{longtable}
\usepackage{multirow}

\usepackage{enumitem}
\usepackage{mathtools}
\usepackage{steinmetz}
\usepackage{tikz}
\usepackage{circuitikz}
\usepackage{verbatim}
\usepackage{tfrupee}
\usepackage[breaklinks=true]{hyperref}

\usepackage{tkz-euclide}

\usetikzlibrary{calc,math}
\usepackage{listings}
    \usepackage{color}                                            %%
    \usepackage{array}                                            %%
    \usepackage{longtable}                                        %%
    \usepackage{calc}                                             %%
    \usepackage{multirow}                                         %%
    \usepackage{hhline}                                           %%
    \usepackage{ifthen}                                           %%
    \usepackage{lscape}     
\usepackage{multicol}
\usepackage{chngcntr}

\DeclareMathOperator*{\Res}{Res}

\renewcommand\thesection{\arabic{section}}
\renewcommand\thesubsection{\thesection.\arabic{subsection}}
\renewcommand\thesubsubsection{\thesubsection.\arabic{subsubsection}}

\renewcommand\thesectiondis{\arabic{section}}
\renewcommand\thesubsectiondis{\thesectiondis.\arabic{subsection}}
\renewcommand\thesubsubsectiondis{\thesubsectiondis.\arabic{subsubsection}}


\hyphenation{op-tical net-works semi-conduc-tor}
\def\inputGnumericTable{}                                 %%

\lstset{
%language=C,
frame=single, 
breaklines=true,
columns=fullflexible
}
\begin{document}


\newtheorem{theorem}{Theorem}[section]
\newtheorem{problem}{Problem}
\newtheorem{proposition}{Proposition}[section]
\newtheorem{lemma}{Lemma}[section]
\newtheorem{corollary}[theorem]{Corollary}
\newtheorem{example}{Example}[section]
\newtheorem{definition}[problem]{Definition}

\newcommand{\BEQA}{\begin{eqnarray}}
\newcommand{\EEQA}{\end{eqnarray}}
\newcommand{\define}{\stackrel{\triangle}{=}}
\bibliographystyle{IEEEtran}
\providecommand{\mbf}{\mathbf}
\providecommand{\pr}[1]{\ensuremath{\Pr\left(#1\right)}}
\providecommand{\qfunc}[1]{\ensuremath{Q\left(#1\right)}}
\providecommand{\sbrak}[1]{\ensuremath{{}\left[#1\right]}}
\providecommand{\lsbrak}[1]{\ensuremath{{}\left[#1\right.}}
\providecommand{\rsbrak}[1]{\ensuremath{{}\left.#1\right]}}
\providecommand{\brak}[1]{\ensuremath{\left(#1\right)}}
\providecommand{\lbrak}[1]{\ensuremath{\left(#1\right.}}
\providecommand{\rbrak}[1]{\ensuremath{\left.#1\right)}}
\providecommand{\cbrak}[1]{\ensuremath{\left\{#1\right\}}}
\providecommand{\lcbrak}[1]{\ensuremath{\left\{#1\right.}}
\providecommand{\rcbrak}[1]{\ensuremath{\left.#1\right\}}}
\theoremstyle{remark}
\newtheorem{rem}{Remark}
\newcommand{\sgn}{\mathop{\mathrm{sgn}}}
\providecommand{\abs}[1]{\left\vert#1\right\vert}
\providecommand{\res}[1]{\Res\displaylimits_{#1}} 
\providecommand{\norm}[1]{\left\lVert#1\right\rVert}
%\providecommand{\norm}[1]{\lVert#1\rVert}
\providecommand{\mtx}[1]{\mathbf{#1}}
\providecommand{\mean}[1]{E\left[ #1 \right]}
\providecommand{\fourier}{\overset{\mathcal{F}}{ \rightleftharpoons}}
%\providecommand{\hilbert}{\overset{\mathcal{H}}{ \rightleftharpoons}}
\providecommand{\system}{\overset{\mathcal{H}}{ \longleftrightarrow}}
	%\newcommand{\solution}[2]{\textbf{Solution:}{#1}}
\newcommand{\solution}{\noindent \textbf{Solution: }}
\newcommand{\cosec}{\,\text{cosec}\,}
\providecommand{\dec}[2]{\ensuremath{\overset{#1}{\underset{#2}{\gtrless}}}}
\newcommand{\myvec}[1]{\ensuremath{\begin{pmatrix}#1\end{pmatrix}}}
\newcommand{\mydet}[1]{\ensuremath{\begin{vmatrix}#1\end{vmatrix}}}
\numberwithin{equation}{subsection}
\makeatletter
\@addtoreset{figure}{problem}
\makeatother
\let\StandardTheFigure\thefigure
\let\vec\mathbf
\renewcommand{\thefigure}{\theproblem}
\def\putbox#1#2#3{\makebox[0in][l]{\makebox[#1][l]{}\raisebox{\baselineskip}[0in][0in]{\raisebox{#2}[0in][0in]{#3}}}}
     \def\rightbox#1{\makebox[0in][r]{#1}}
     \def\centbox#1{\makebox[0in]{#1}}
     \def\topbox#1{\raisebox{-\baselineskip}[0in][0in]{#1}}
     \def\midbox#1{\raisebox{-0.5\baselineskip}[0in][0in]{#1}}
\vspace{3cm}
\title{EE5609 Assignment 9}
\author{Abhishek Thakur}
\maketitle
\newpage
\bigskip
\renewcommand{\thefigure}{\theenumi}
\renewcommand{\thetable}{\theenumi}
\begin{abstract}
This document solves problem based on solution of vector space.
\end{abstract}
Download all solutions from 
\begin{lstlisting}
https://github.com/abhishekt711/EE5609/tree/master/Assignment_9
\end{lstlisting}
\section{Problem}
let $\alpha=\myvec{x_1\\x_2}$ and $\beta=\myvec{y_1\\y_2}$ be vectors in $\mathbb{R}^2$ such that\\
$x_1y_1+x_2y_2=0$;\quad $x_1^2+x_2^2=y_1^2+y_2^2=1$.\\
Proove that $\ss=\{\vec{\alpha},\vec{\beta}\}$ is a basis of $\mathbb{R}^2$. Find the coordinates of the vector $\myvec{a,b}$ in the ordered basis $\ss=\{\vec{\alpha},\vec{\beta}\}$. (The conditions on $\alpha$ and $\beta$ say, geometrically, that $\alpha$ and $\beta$ are perpendicular and each has length 1).
\section{Solution}
we need to show that $\alpha$ and $\beta$ are linearly independent in order to proove that $\ss=\{\vec{\alpha},\vec{\beta}\}$ is a basis of $\mathbb{R}^2$.\\


Given in the question are:
\begin{align}
\alpha^T\beta&=0\label{eq0}\\
\norm{\alpha}^2 &=\norm{\beta}^2=1 \label{eq1}
\end{align}
Let,
\begin{align}
 \vec{A}=\myvec{\alpha & \beta}=\myvec{x_1&y_1\\x_2&y_2}
\end{align}
then,
\begin{align}
 \vec{A}^T\vec{A}&=\myvec{\norm{\alpha}^2 &\alpha^T\beta\\\alpha^T\beta&\norm{\beta}^2}\\ &=\myvec{1&0\\0&1}\\
 \therefore \vec{A}^T\vec{A}&=\vec{I}
\end{align}
Inverse of $\vec{A}$ exist. $\vec{A}^T$ is the inverse of $\vec{A}$. 
Thus, the columns of $\vec{A}$ are linearly independent i.e, $\alpha$ and $\beta$ are linearly independent. \\
Hence, $\ss=\{\vec{\alpha},\vec{\beta}\}$ is a basis of $\mathbb{R}^2$.\\



To, find the coordinates of the vector $\myvec{a,b}$ in the ordered basis $\ss=\{\vec{\alpha},\vec{\beta}\}$. We
can row-reduce the augmented matrix,
   \begin{align}
      \myvec{x_1&y_1&\vrule&a\\x_2&y_2&\vrule&b}\\
        \xleftrightarrow{R_1=\frac{R_1}{x_1}}\myvec{1&\frac{y_1}{x_1}&\vrule&\frac{a}{x_1}\\x_2&y_2&\vrule&b}\\
     \xleftrightarrow{R_2=R_2-x_2R_1}\myvec{1&\frac{y_1}{x_1}&\vrule&\frac{a}{x_1}\\0&y_2-\frac{x_2y_1}{x_1}&\vrule&b-\frac{x_2a}{x_1}}\\
     =\myvec{1&\frac{y_1}{x_1}&\vrule&\frac{a}{x_1}\\0&\frac{x_1y_2-x_2y_1}{x_1}&\vrule&\frac{x_1b-x_2a}{x_1}}\\
      \xleftrightarrow{R_2=\left(\frac{x_1}{x_1y_2-x_2y_1}\right)R_2}\myvec{1&\frac{y_1}{x_1}&\vrule&\frac{a}{x_1}\\0&1&\vrule&\frac{x_1b-x_2a}{x_1y_2-x_2y_1}}\\
      \xleftrightarrow{R_1=R_1-\left(\frac{y_1}{x_1}\right)R_2}\myvec{1&0&\vrule&\frac{ay_2-by_1}{x_1y_2-x_2y_1}\\0&1&\vrule&\frac{x_1b-x_2a}{x_1y_2-x_2y_1}}\label{eq12}
\end{align}
Using \ref{eq0} and \ref{eq1} and simplifying in \ref{eq12},
\begin{align}
=\myvec{1&0&\vrule&ax_1+bx_2\\0&1&\vrule&ay_1+by_2}
\end{align}
Hence,
\begin{align}
\implies \myvec{a\\b}&=\myvec{\alpha & \beta}\myvec{ax_1+bx_2\\ay_1+by_2}\\
 \therefore \myvec{a\\b}&=\myvec{x_1&y_1 \\ x_2&y_2}\myvec{ax_1+bx_2\\ay_1+by_2}
\end{align}

\end{document}