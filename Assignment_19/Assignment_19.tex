                                                                                                                                                                                                                                                                                                                                                                                                                                                                                                                                                                                                                                                                                \documentclass[journal,12pt,twocolumn]{IEEEtran}

\usepackage{setspace}
\usepackage{gensymb}

\singlespacing


\usepackage[cmex10]{amsmath}

\usepackage{amsthm}

\usepackage{mathrsfs}
\usepackage{txfonts}
\usepackage{stfloats}
\usepackage{bm}
\usepackage{cite}
\usepackage{cases}
\usepackage{subfig}

\usepackage{longtable}
\usepackage{multirow}

\usepackage{enumitem}
\usepackage{mathtools}
\usepackage{steinmetz}
\usepackage{tikz}
\usepackage{circuitikz}
\usepackage{verbatim}
\usepackage{tfrupee}
\usepackage[breaklinks=true]{hyperref}

\usepackage{tkz-euclide}

\usetikzlibrary{calc,math}
\usepackage{listings}
    \usepackage{color}                                            %%
    \usepackage{array}                                            %%
    \usepackage{longtable}                                        %%
    \usepackage{calc}                                             %%
    \usepackage{multirow}                                         %%
    \usepackage{hhline}                                           %%
    \usepackage{ifthen}                                           %%
    \usepackage{lscape}     
\usepackage{multicol}
\usepackage{chngcntr}
\usepackage{graphicx}
\DeclareMathOperator*{\Res}{Res}

\renewcommand\thesection{\arabic{section}}
\renewcommand\thesubsection{\thesection.\arabic{subsection}}
\renewcommand\thesubsubsection{\thesubsection.\arabic{subsubsection}}

\renewcommand\thesectiondis{\arabic{section}}
\renewcommand\thesubsectiondis{\thesectiondis.\arabic{subsection}}
\renewcommand\thesubsubsectiondis{\thesubsectiondis.\arabic{subsubsection}}


\hyphenation{op-tical net-works semi-conduc-tor}
\def\inputGnumericTable{}                                 %%

\lstset{
%language=C,
frame=single, 
breaklines=true,
columns=fullflexible
}
\begin{document}


\newtheorem{theorem}{Theorem}[section]
\newtheorem{problem}{Problem}
\newtheorem{proposition}{Proposition}[section]
\newtheorem{lemma}{Lemma}[section]
\newtheorem{corollary}[theorem]{Corollary}
\newtheorem{example}{Example}[section]
\newtheorem{definition}[problem]{Definition}

\newcommand{\BEQA}{\begin{eqnarray}}
\newcommand{\EEQA}{\end{eqnarray}}
\newcommand{\define}{\stackrel{\triangle}{=}}
\bibliographystyle{IEEEtran}
\providecommand{\mbf}{\mathbf}
\providecommand{\pr}[1]{\ensuremath{\Pr\left(#1\right)}}
\providecommand{\qfunc}[1]{\ensuremath{Q\left(#1\right)}}
\providecommand{\sbrak}[1]{\ensuremath{{}\left[#1\right]}}
\providecommand{\lsbrak}[1]{\ensuremath{{}\left[#1\right.}}
\providecommand{\rsbrak}[1]{\ensuremath{{}\left.#1\right]}}
\providecommand{\brak}[1]{\ensuremath{\left(#1\right)}}
\providecommand{\lbrak}[1]{\ensuremath{\left(#1\right.}}
\providecommand{\rbrak}[1]{\ensuremath{\left.#1\right)}}
\providecommand{\cbrak}[1]{\ensuremath{\left\{#1\right\}}}
\providecommand{\lcbrak}[1]{\ensuremath{\left\{#1\right.}}
\providecommand{\rcbrak}[1]{\ensuremath{\left.#1\right\}}}
\theoremstyle{remark}
\newtheorem{rem}{Remark}
\newcommand{\sgn}{\mathop{\mathrm{sgn}}}
\providecommand{\abs}[1]{\left\vert#1\right\vert}
\providecommand{\res}[1]{\Res\displaylimits_{#1}} 
\providecommand{\norm}[1]{\left\lVert#1\right\rVert}
%\providecommand{\norm}[1]{\lVert#1\rVert}
\providecommand{\mtx}[1]{\mathbf{#1}}
\providecommand{\mean}[1]{E\left[ #1 \right]}
\providecommand{\fourier}{\overset{\mathcal{F}}{ \rightleftharpoons}}
%\providecommand{\hilbert}{\overset{\mathcal{H}}{ \rightleftharpoons}}
\providecommand{\system}{\overset{\mathcal{H}}{ \longleftrightarrow}}
	%\newcommand{\solution}[2]{\textbf{Solution:}{#1}}
\newcommand{\solution}{\noindent \textbf{Solution: }}
\newcommand{\cosec}{\,\text{cosec}\,}
\providecommand{\dec}[2]{\ensuremath{\overset{#1}{\underset{#2}{\gtrless}}}}
\newcommand{\myvec}[1]{\ensuremath{\begin{pmatrix}#1\end{pmatrix}}}
\newcommand{\mydet}[1]{\ensuremath{\begin{vmatrix}#1\end{vmatrix}}}
\numberwithin{equation}{subsection}
\makeatletter
\@addtoreset{figure}{problem}
\makeatother
\let\StandardTheFigure\thefigure
\let\vec\mathbf
\renewcommand{\thefigure}{\theproblem}
\def\putbox#1#2#3{\makebox[0in][l]{\makebox[#1][l]{}\raisebox{\baselineskip}[0in][0in]{\raisebox{#2}[0in][0in]{#3}}}}
     \def\rightbox#1{\makebox[0in][r]{#1}}
     \def\centbox#1{\makebox[0in]{#1}}
     \def\topbox#1{\raisebox{-\baselineskip}[0in][0in]{#1}}
     \def\midbox#1{\raisebox{-0.5\baselineskip}[0in][0in]{#1}}
\vspace{3cm}
\title{EE5609 Assignment 19}
\author{Abhishek Thakur}
\maketitle
\bigskip
\renewcommand{\thefigure}{\theenumi}
\renewcommand{\thetable}{\theenumi}
\renewcommand{\labelenumi}{\alph{enumi})}
\begin{abstract}
This document solves problem based on Matrix Theory.
\end{abstract}
Download all solutions from 
\begin{lstlisting}
https://github.com/abhishekt711/EE5609/tree/master/Assignment_19
\end{lstlisting}
\section{Problem}
Let $\myvec{1&-1&1\\2&1&4\\-2&1&-4}$, Given that $1$ is an eigenvalue of M, then which among the following are correct?
\begin{enumerate}
\item The minimal polynomial of $M$ is $(X-1)(X+4)$.
\item The minimal polynomial of $M$ is $(X-1)^2(X+4)$.
\item $M$ is not diagonalizable
\item $M^{-1}=\frac{1}{4}(M+3I)$.
\end{enumerate}
\section{solution}
Given,
\begin{align}
\vec{M}=\myvec{1&-1&1\\2&1&4\\-2&1&-4}\\
\lambda_1=1
\end{align}
We need to find the other eigenvalues,
\begin{align}
\lvert M-\lambda I \rvert =0\\
\myvec{1-\lambda&-1 &1\\2&1-\lambda&4\\-2&1&-4-\lambda}=0\\
\lambda^3+2\lambda^2-7\lambda+4=0\\
\implies (\lambda-1)^2(\lambda+4)=0\\
\therefore \lambda=1 , -4
\end{align}
For $\lambda=1$, corresponding eigenvector is $\myvec{-2\\1\\1}$.\\
For $\lambda=-4$, corresponding eigenvector is $\myvec{-\frac{1}{3}\\-\frac{2}{3}\\1}$.\\
For $3\times3$ matrix only $2$ independent eigenvectors are there. Hence, $P^{-1}$ does not exist and $M$ can't be diagonalizable.\\
Thus, option c is correct.\\
Characteristic equation is given as,
\begin{align}
p(x)=(X-1)^2(X+4)
\end{align}
Minimal polynomial will be,
\begin{align}
p(x)=(X-1)^a(X+4)^b : a \leq 2, b \leq 1
\end{align}
For $a=1$, $b=1$,
\begin{align}
p(\vec{M})=(\vec{M}-1)(\vec{M}+4)\neq0
\end{align}
$\therefore (X-1)(X+4)$ is not a minimal polynomial.\\
Thus option a is not correct.\\
For $a=2$, $b=1$,
\begin{align}
p(\vec{M})=(\vec{M}-1)^2(\vec{M}+4)=0
\end{align}
$\therefore (X-1)^2(X+4)$ is minimal polynomial.\\
Thus option b is correct
\begin{align}
p(\vec{M})=(\vec{M}-1)^2(\vec{M}+4)=0\\
\implies \vec{M}^3+2\vec{M}^2-7\vec{M}+4\vec{I}=0\\
\vec{I}= -\frac{1}{4}(\vec{M}^3+2\vec{M}^2-7\vec{M})\\
\vec{M}^{-1}=-\frac{1}{4}(\vec{M}^2+2\vec{M}-7\vec{I})
\end{align}
Thus option d is not correct.\\
Hence, the correct options are b and c.
\end{document}
