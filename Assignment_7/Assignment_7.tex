\documentclass[journal,12pt,twocolumn]{IEEEtran}

\usepackage{setspace}
\usepackage{gensymb}

\singlespacing


\usepackage[cmex10]{amsmath}

\usepackage{amsthm}

\usepackage{mathrsfs}
\usepackage{txfonts}
\usepackage{stfloats}
\usepackage{bm}
\usepackage{cite}
\usepackage{cases}
\usepackage{subfig}

\usepackage{longtable}
\usepackage{multirow}

\usepackage{enumitem}
\usepackage{mathtools}
\usepackage{steinmetz}
\usepackage{tikz}
\usepackage{circuitikz}
\usepackage{verbatim}
\usepackage{tfrupee}
\usepackage[breaklinks=true]{hyperref}

\usepackage{tkz-euclide}

\usetikzlibrary{calc,math}
\usepackage{listings}
    \usepackage{color}                                            %%
    \usepackage{array}                                            %%
    \usepackage{longtable}                                        %%
    \usepackage{calc}                                             %%
    \usepackage{multirow}                                         %%
    \usepackage{hhline}                                           %%
    \usepackage{ifthen}                                           %%
    \usepackage{lscape}     
\usepackage{multicol}
\usepackage{chngcntr}

\DeclareMathOperator*{\Res}{Res}

\renewcommand\thesection{\arabic{section}}
\renewcommand\thesubsection{\thesection.\arabic{subsection}}
\renewcommand\thesubsubsection{\thesubsection.\arabic{subsubsection}}

\renewcommand\thesectiondis{\arabic{section}}
\renewcommand\thesubsectiondis{\thesectiondis.\arabic{subsection}}
\renewcommand\thesubsubsectiondis{\thesubsectiondis.\arabic{subsubsection}}


\hyphenation{op-tical net-works semi-conduc-tor}
\def\inputGnumericTable{}                                 %%

\lstset{
%language=C,
frame=single, 
breaklines=true,
columns=fullflexible
}
\begin{document}


\newtheorem{theorem}{Theorem}[section]
\newtheorem{problem}{Problem}
\newtheorem{proposition}{Proposition}[section]
\newtheorem{lemma}{Lemma}[section]
\newtheorem{corollary}[theorem]{Corollary}
\newtheorem{example}{Example}[section]
\newtheorem{definition}[problem]{Definition}

\newcommand{\BEQA}{\begin{eqnarray}}
\newcommand{\EEQA}{\end{eqnarray}}
\newcommand{\define}{\stackrel{\triangle}{=}}
\bibliographystyle{IEEEtran}
\providecommand{\mbf}{\mathbf}
\providecommand{\pr}[1]{\ensuremath{\Pr\left(#1\right)}}
\providecommand{\qfunc}[1]{\ensuremath{Q\left(#1\right)}}
\providecommand{\sbrak}[1]{\ensuremath{{}\left[#1\right]}}
\providecommand{\lsbrak}[1]{\ensuremath{{}\left[#1\right.}}
\providecommand{\rsbrak}[1]{\ensuremath{{}\left.#1\right]}}
\providecommand{\brak}[1]{\ensuremath{\left(#1\right)}}
\providecommand{\lbrak}[1]{\ensuremath{\left(#1\right.}}
\providecommand{\rbrak}[1]{\ensuremath{\left.#1\right)}}
\providecommand{\cbrak}[1]{\ensuremath{\left\{#1\right\}}}
\providecommand{\lcbrak}[1]{\ensuremath{\left\{#1\right.}}
\providecommand{\rcbrak}[1]{\ensuremath{\left.#1\right\}}}
\theoremstyle{remark}
\newtheorem{rem}{Remark}
\newcommand{\sgn}{\mathop{\mathrm{sgn}}}
\providecommand{\abs}[1]{\left\vert#1\right\vert}
\providecommand{\res}[1]{\Res\displaylimits_{#1}} 
\providecommand{\norm}[1]{\left\lVert#1\right\rVert}
%\providecommand{\norm}[1]{\lVert#1\rVert}
\providecommand{\mtx}[1]{\mathbf{#1}}
\providecommand{\mean}[1]{E\left[ #1 \right]}
\providecommand{\fourier}{\overset{\mathcal{F}}{ \rightleftharpoons}}
%\providecommand{\hilbert}{\overset{\mathcal{H}}{ \rightleftharpoons}}
\providecommand{\system}{\overset{\mathcal{H}}{ \longleftrightarrow}}
	%\newcommand{\solution}[2]{\textbf{Solution:}{#1}}
\newcommand{\solution}{\noindent \textbf{Solution: }}
\newcommand{\cosec}{\,\text{cosec}\,}
\providecommand{\dec}[2]{\ensuremath{\overset{#1}{\underset{#2}{\gtrless}}}}
\newcommand{\myvec}[1]{\ensuremath{\begin{pmatrix}#1\end{pmatrix}}}
\newcommand{\mydet}[1]{\ensuremath{\begin{vmatrix}#1\end{vmatrix}}}
\numberwithin{equation}{subsection}
\makeatletter
\@addtoreset{figure}{problem}
\makeatother
\let\StandardTheFigure\thefigure
\let\vec\mathbf
\renewcommand{\thefigure}{\theproblem}
\def\putbox#1#2#3{\makebox[0in][l]{\makebox[#1][l]{}\raisebox{\baselineskip}[0in][0in]{\raisebox{#2}[0in][0in]{#3}}}}
     \def\rightbox#1{\makebox[0in][r]{#1}}
     \def\centbox#1{\makebox[0in]{#1}}
     \def\topbox#1{\raisebox{-\baselineskip}[0in][0in]{#1}}
     \def\midbox#1{\raisebox{-0.5\baselineskip}[0in][0in]{#1}}
\vspace{3cm}
\title{EE5609 Assignment 7}
\author{Abhishek Thakur}
\maketitle
\newpage
\bigskip
\renewcommand{\thefigure}{\theenumi}
\renewcommand{\thetable}{\theenumi}
\begin{abstract}
This document solves problem based on Singular Value Decomposition.
\end{abstract}
\vspace{0.5cm}
%
Download all solutions from 
\begin{lstlisting}
https://github.com/abhishekt711/EE5609/tree/master/Assignment_7 
\end{lstlisting}
%
%
\vspace{0.5mm}
\section{Problem}
Find the foot of the perpendicular from
\begin{align}
\vec{c}=\myvec{3\\3\\5}
\end{align}
to the given plane
\begin{align}
2x+3y-4z+5=0 
\end{align}
\section{Solution}
The given equation of plane can be represented as
\begin{align}
    \myvec{2 & 3 & -4}\Vec{x}=-5\label{eq1}\\
    \vec{n}=\myvec{2\\3\\-4}
\end{align}
We need to find two vectors $\vec{m_1}$ and $\Vec{m_2}$ that are $\perp$ to $\vec{n}$
\begin{align}\label{eq2}
	\implies\myvec{2 & 3 & -4}\myvec{a\\b\\c} = 0
\end{align}
Put $a=1$ and $b=0$ in \eqref{eq2},we get,
\begin{align}
    \Vec{m_1}=\myvec{1\\0\\\frac{1}{2}}
\end{align}
Put $a=0$ and $b=1$ in \eqref{eq2},we get,
\begin{align}
    \Vec{m_2}=\myvec{0\\1\\\frac{3}{4}}
\end{align}
Now, solving the equation
\begin{align}
	\label{eq3}\vec{M}\Vec{x} &= \vec{c}
\end{align}
where,
\begin{align}
\Vec{M}=\myvec{1&0\\0&1\\\frac{1}{2}&\frac{3}{4}}&\\
    \vec{c}=\myvec{3\\3\\5}&
\end{align}
Now, to solve equation \eqref{eq3}, we perform Singular Value Decomposition on $\vec{M}$ as follows,
\begin{align}
\vec{M}=\vec{U}\vec{S}\vec{V}^T\label{eqSVD}
\end{align}
Substituting the value of $\Vec{M}$ from equation \eqref{eqSVD} to\eqref{eq3},
\begin{align}
    \vec{U}\vec{S}\vec{V}^T\Vec{x}=\vec{c}\\
    \implies \vec{x}=\vec{V}\vec{S_+}\vec{U}^T\Vec{c} \label{eqX}
\end{align}
Where, $\Vec{S_+}$ is the Moore-Pen-rose Pseudo-Inverse of $\Vec{S}$. Columns of $\vec{V}$ are the eigen vectors of $\vec{M}^T\vec{M}$, columns of $\vec{U}$ are the eigen vectors of $\vec{M}\vec{M}^T$ and $\vec{S}$ is diagonal matrix of singular value of eigenvalues of $\vec{M}^T\vec{M}$.
\begin{align}
\vec{M}^T\vec{M}=\myvec{\frac{5}{4}&\frac{3}{8}\\\frac{3}{8}&\frac{25}{16}}\label{eqMTM}
\end{align}
Eigen values corresponding to $\Vec{M}^T\Vec{M}$ are given by,
\begin{align}
\mydet{\vec{M}^T\vec{M}-\lambda\vec{I}} &= 0\\
\implies\mydet{\myvec{\frac{5}{4}-\lambda&\frac{3}{8}\\\frac{3}{8}&\frac{25}{16}-\lambda}} &=0\\
\implies \lambda^2-\frac{45}{16}\lambda+\frac{29}{16}=0
\end{align}
Hence eigen values of $\vec{M}^T\vec{M}$ are,
\begin{align}
\lambda_1 &= \frac{29}{16}\\
\lambda_2 &= 1
\end{align}
Hence the eigen vectors of $\vec{M}^T\vec{M}$ are,
\begin{align}
\vec{v}_1=\myvec{\frac{2}{3}\\1}\\
\vec{v}_2=\myvec{-\frac{3}{2}\\1}
\end{align}
Normalizing the eigen vectors, we obtain $\vec{V}$ of \eqref{eqSVD} as follows,
\begin{align}
\vec{V}=\myvec{\frac{2}{\sqrt{13}}&-\frac{3}{\sqrt{13}}\\ \frac{3}{\sqrt{13}}&\frac{2}{\sqrt{13}}}\label{eqV}
\end{align}
$\vec{S}$ of the diagonal matrix of \eqref{eqSVD} is:
\begin{align}
\vec{S}=\myvec{\frac{\sqrt{29}}{4}&0\\0&1\\0&0}\label{eqS}
\end{align}
Now, calculating eigen value of $\vec{M}\vec{M}^T$,
\begin{align}
    \Vec{M}\Vec{M}^T=\myvec{1&0&\frac{1}{2}\\0&1&\frac{3}{4}\\\frac{1}{2}&\frac{3}{4}&\frac{13}{16}}
\end{align}
Eigen values corresponding to $\vec{M}\Vec{M}^T$ are given by
\begin{align}
\mydet{\vec{M}\vec{M}^T-\lambda\vec{I}} &= 0\\
\implies \mydet{\myvec{1-\lambda&0&\frac{1}{2}\\0&1-\lambda&\frac{3}{4}\\\frac{1}{2}&\frac{3}{4}&\frac{13}{16}-\lambda}}=0\\
\implies \lambda^3-\frac{45}{16}\lambda^2+\frac{29}{16}\lambda=0
\end{align}
Hence eigen values of $\vec{M}^T\vec{M}$ are,
\begin{align}
\lambda_3 &=\frac{29}{16}\\
\lambda_4 &= 1\\
\lambda_5 &= 0
\end{align}
Hence we obtain $\vec{U}$ of \eqref{eqSVD} as follows,
\begin{align}
\vec{U}=\label{eqU}\myvec{\frac{8}{\sqrt{377}}&-\frac{3}{\sqrt{13}}&-\frac{2}{\sqrt{29}}\\\frac{12}{\sqrt{377}}&\frac{2}{\sqrt{13}}&-\frac{3}{\sqrt{29}}\\\frac{13}{\sqrt{377}}&0&\frac{4}{\sqrt{29}}}
\end{align}
Finally from \eqref{eqSVD} we get the Singular Value Decomposition of $\vec{M}$ as follows,
\begin{align}
\vec{M} = \myvec{\frac{8}{\sqrt{377}}&-\frac{3}{\sqrt{13}}&-\frac{2}{\sqrt{29}}\\\frac{12}{\sqrt{377}}&\frac{2}{\sqrt{13}}&-\frac{3}{\sqrt{29}}\\\frac{13}{\sqrt{377}}&0&\frac{4}{\sqrt{29}}}\myvec{\frac{\sqrt{29}}{4}&0\\0&1\\0&0}\myvec{\frac{2}{\sqrt{13}}&-\frac{3}{\sqrt{13}}\\ \frac{3}{\sqrt{13}}&\frac{2}{\sqrt{13}}}^T
\end{align}
Now, Moore-Penrose Pseudo inverse of $\vec{S}$ is given by,
\begin{align}
\vec{S_+} = \myvec{\frac{4}{\sqrt{29}}&0&0\\0&1&0}\label{eqS+}
\end{align}
Substituting the values of \eqref{eqU},\eqref{eqV},\eqref{eqS+} in \eqref{eqX}  we get,
\begin{align}
\vec{U}^T\vec{c}&=\myvec{\frac{125}{\sqrt{377}}\\\frac{-3}{\sqrt{13}}\\\frac{5}{\sqrt{29}}}\\
\vec{S_+}\vec{U}^T\vec{c}&=\myvec{\frac{500}{29\sqrt{13}}\\-\frac{3}{\sqrt{13}}}\\
\vec{x} = \vec{V}\vec{S_+}\vec{U}^T\vec{c} &= \myvec{\frac{97}{29}\\\frac{102}{29}}\label{eqXSol1}
\end{align}
Verifying the solution of \eqref{eqXSol1} using,
\begin{align}
\implies \vec{M}^T\vec{M}\vec{x} = \vec{M}^T\vec{c}\label{eqVerify}
\end{align}
Evaluating the R.H.S in \eqref{eqVerify} we get,
\begin{align}
\vec{M}^T\vec{c} &= \myvec{1&0&\frac{1}{2}\\0&1&\frac{3}{4}}\myvec{3\\3\\5}=\myvec{\frac{11}{2}\\\frac{27}{4}}\\
&\implies\myvec{\frac{5}{4}&\frac{3}{8}\\\frac{3}{8}&\frac{25}{16}}\vec{x} = \myvec{\frac{11}{2}\\\frac{27}{4}}&\label{eqMateq}
\end{align}
The augmented matrix of \eqref{eqMateq} is,
\begin{align}
\myvec{\frac{5}{4}&\frac{3}{8} &\frac{11}{2}\\\frac{3}{8}&\frac{25}{16}&\frac{27}{4}} & \label{eqAugment}
\end{align}
Solving the augmented matrix into Row reduced echelon form of \eqref{eqAugment} we get,
\begin{align}
\myvec{\frac{5}{4}&\frac{3}{8} &\frac{11}{2}\\\frac{3}{8}&\frac{25}{16}&\frac{27}{4}}\xleftrightarrow[]{R_1\leftarrow\frac{4}{5}R_1}\myvec{1&\frac{3}{10}&\frac{22}{5}\\\frac{3}{8}&\frac{25}{16}&\frac{27}{4}}\\
\xleftrightarrow[]{R_2\leftarrow R_2-\frac{3}{8}R_1}\myvec{1&\frac{3}{10}&\frac{22}{5}\\0&\frac{29}{20}&\frac{51}{10}}\\
\xleftrightarrow[]{R_2\leftarrow \frac{20}{29}R_2}\myvec{1&\frac{3}{10}&\frac{22}{5}\\0&1&\frac{102}{29}}\\
\xleftrightarrow[]{R_1\leftarrow R_1-\frac{3}{10}R_2}\myvec{1&0&\frac{97}{29}\\0&1&\frac{102}{29}}
\end{align}
Therefore,
\begin{align}
   \vec{x}&=\myvec{\frac{97}{29}\\\frac{102}{29}}\label{eqrref}
\end{align}
Comparing results of $\vec{x}$ from \eqref{eqXSol1} and \eqref{eqrref}, Hence, the solution is verified.
\end{document}