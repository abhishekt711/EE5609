                                                                                                                                                                                                                                                                                                                                                                                                                                                                                                                                                                                                                                                                                \documentclass[journal,12pt,onecolumn]{IEEEtran}

\usepackage{setspace}
\usepackage{gensymb}

\singlespacing


\usepackage[cmex10]{amsmath}

\usepackage{amsthm}

\usepackage{mathrsfs}
\usepackage{txfonts}
\usepackage{stfloats}
\usepackage{bm}
\usepackage{cite}
\usepackage{cases}
\usepackage{subfig}

\usepackage{longtable}
\usepackage{multirow}

\usepackage{enumitem}
\usepackage{mathtools}
\usepackage{steinmetz}
\usepackage{tikz}
\usepackage{circuitikz}
\usepackage{verbatim}
\usepackage{tfrupee}
\usepackage[breaklinks=true]{hyperref}

\usepackage{tkz-euclide}

\usetikzlibrary{calc,math}
\usepackage{listings}
    \usepackage{color}                                            %%
    \usepackage{array}                                            %%
    \usepackage{longtable}                                        %%
    \usepackage{calc}                                             %%
    \usepackage{multirow}                                         %%
    \usepackage{hhline}                                           %%
    \usepackage{ifthen}                                           %%
    \usepackage{lscape}     
\usepackage{multicol}
\usepackage{chngcntr}
\usepackage{graphicx}
\DeclareMathOperator*{\Res}{Res}

\renewcommand\thesection{\arabic{section}}
\renewcommand\thesubsection{\thesection.\arabic{subsection}}
\renewcommand\thesubsubsection{\thesubsection.\arabic{subsubsection}}

\renewcommand\thesectiondis{\arabic{section}}
\renewcommand\thesubsectiondis{\thesectiondis.\arabic{subsection}}
\renewcommand\thesubsubsectiondis{\thesubsectiondis.\arabic{subsubsection}}


\hyphenation{op-tical net-works semi-conduc-tor}
\def\inputGnumericTable{}                                 %%

\lstset{
%language=C,
frame=single, 
breaklines=true,
columns=fullflexible
}
\begin{document}


\newtheorem{theorem}{Theorem}[section]
\newtheorem{problem}{Problem}
\newtheorem{proposition}{Proposition}[section]
\newtheorem{lemma}{Lemma}[section]
\newtheorem{corollary}[theorem]{Corollary}
\newtheorem{example}{Example}[section]
\newtheorem{definition}[problem]{Definition}

\newcommand{\BEQA}{\begin{eqnarray}}
\newcommand{\EEQA}{\end{eqnarray}}
\newcommand{\define}{\stackrel{\triangle}{=}}
\bibliographystyle{IEEEtran}
\providecommand{\mbf}{\mathbf}
\providecommand{\pr}[1]{\ensuremath{\Pr\left(#1\right)}}
\providecommand{\qfunc}[1]{\ensuremath{Q\left(#1\right)}}
\providecommand{\sbrak}[1]{\ensuremath{{}\left[#1\right]}}
\providecommand{\lsbrak}[1]{\ensuremath{{}\left[#1\right.}}
\providecommand{\rsbrak}[1]{\ensuremath{{}\left.#1\right]}}
\providecommand{\brak}[1]{\ensuremath{\left(#1\right)}}
\providecommand{\lbrak}[1]{\ensuremath{\left(#1\right.}}
\providecommand{\rbrak}[1]{\ensuremath{\left.#1\right)}}
\providecommand{\cbrak}[1]{\ensuremath{\left\{#1\right\}}}
\providecommand{\lcbrak}[1]{\ensuremath{\left\{#1\right.}}
\providecommand{\rcbrak}[1]{\ensuremath{\left.#1\right\}}}
\theoremstyle{remark}
\newtheorem{rem}{Remark}
\newcommand{\sgn}{\mathop{\mathrm{sgn}}}
\providecommand{\abs}[1]{\left\vert#1\right\vert}
\providecommand{\res}[1]{\Res\displaylimits_{#1}} 
\providecommand{\norm}[1]{\left\lVert#1\right\rVert}
%\providecommand{\norm}[1]{\lVert#1\rVert}
\providecommand{\mtx}[1]{\mathbf{#1}}
\providecommand{\mean}[1]{E\left[ #1 \right]}
\providecommand{\fourier}{\overset{\mathcal{F}}{ \rightleftharpoons}}
%\providecommand{\hilbert}{\overset{\mathcal{H}}{ \rightleftharpoons}}
\providecommand{\system}{\overset{\mathcal{H}}{ \longleftrightarrow}}
	%\newcommand{\solution}[2]{\textbf{Solution:}{#1}}
\newcommand{\solution}{\noindent \textbf{Solution: }}
\newcommand{\cosec}{\,\text{cosec}\,}
\providecommand{\dec}[2]{\ensuremath{\overset{#1}{\underset{#2}{\gtrless}}}}
\newcommand{\myvec}[1]{\ensuremath{\begin{pmatrix}#1\end{pmatrix}}}
\newcommand{\mydet}[1]{\ensuremath{\begin{vmatrix}#1\end{vmatrix}}}
\numberwithin{equation}{subsection}
\makeatletter
\@addtoreset{figure}{problem}
\makeatother
\let\StandardTheFigure\thefigure
\let\vec\mathbf
\renewcommand{\thefigure}{\theproblem}
\def\putbox#1#2#3{\makebox[0in][l]{\makebox[#1][l]{}\raisebox{\baselineskip}[0in][0in]{\raisebox{#2}[0in][0in]{#3}}}}
     \def\rightbox#1{\makebox[0in][r]{#1}}
     \def\centbox#1{\makebox[0in]{#1}}
     \def\topbox#1{\raisebox{-\baselineskip}[0in][0in]{#1}}
     \def\midbox#1{\raisebox{-0.5\baselineskip}[0in][0in]{#1}}
\vspace{3cm}
\title{EE5609 Assignment 18}
\author{Abhishek Thakur}
\maketitle
\bigskip
\renewcommand{\thefigure}{\theenumi}
\renewcommand{\thetable}{\theenumi}
\renewcommand{\labelenumi}{\alph{enumi})}
\begin{abstract}
This document solves problem based on Matrix Theory.
\end{abstract}
Download all solutions from 
\begin{lstlisting}
https://github.com/abhishekt711/EE5609/tree/master/Assignment_18
\end{lstlisting}
\section{Problem}
Let $T$ be the linear operator on $R^2$, the matrix of which in the standard ordered basis is \\
$\myvec{2&1\\0&2}$\\
Let $W_1$ be the subspace of $R^2$ spanned by the vector $\epsilon_1=(1,0)$
\begin{enumerate}
\item Proove that $W_1$ is invariant under $T$.
\item Prove that there is no subspace $W_2$ which is invariant under $T$ and is complementary to $W_1$:\\
$R^2=W_1\oplus W_2$\\
(Compare with exercise 1 of section 6.5.)
\end{enumerate}
\section{solution}
\begin{longtable}{|p{5cm}|p{13cm}|}
\hline
\textbf{Statement} &\textbf{Solution}\\
\hline
& \\ 
Given
& \parbox{12cm}{$T$ be the linear operator on $R^2$
the matrix of which in the standard ordered basis is $\myvec{2&1\\0&2}$ \\
$W_1$ be the subspace of $R^2$ spanned by the vector $\epsilon_1=(1,0)$.\\}\\
\hline
& \\ 
To Proove
& \parbox{12cm}{\begin{enumerate}
\item {$W_1$ is invariant under $T$.}
\item{There is no subspace $W_2$ which is invariant under $T$ and is complementary to $W_1$:
$R^2=W_1\oplus W_2$}
\item{Compare with exercise 1 of section 6.5.}
\end{enumerate}}
\\
& \\
\hline
Proof (a) &
\parbox{12cm}{\begin{align}
    \vec{A}=\myvec{2&1\\0&2}\\
    \lvert A -\lambda I \rvert=0\\
      \implies \myvec{2-\lambda & 1\\0& 2- \lambda} \\
    = (2-\lambda)^2=0\\
    \therefore \lambda =2
    \end{align}
    for $\lambda = 2$ , the corresponding vector is
    \begin{align}
    (\vec{A}-\lambda I)X=0\\
    \myvec{0&1\\0&0}X=0\\
    \therefore X=\myvec{1\\0}
    \end{align}
    Hence, $W_1$ be the subspace of $R^2$ spanned by the vector $\epsilon_1=(1,0)$ is invariant under $T$.\\
}
\\
\hline
& \\
Proof (b) &
\parbox{12cm}{ Corresponding to $\lambda=2$, \\ Among, two eigen vectors only one is independent and other one is dependent.\\
 Thus, $P^-1$ does not exist and $A$ can not be diagonalized.\\ Hence, there is no subspace $W_2$ which is invariant under $T$ and is complementary to $W_1$:\\
$R^2=W_1\oplus W_2$\\}\\
\hline
& \\
Observation & 
\parbox{12cm}{In exercise 1 of section 6.5, for $2\times 2$ matrix there is 2 distinct characteristic value, corresponding to which there is a eigen vector. Hence, $P^{-1}$ exists. \\
$\therefore$ the given matrix is diagonalizable.\\}\\
\hline
\caption*{Table1:Solution}
\end{longtable}
\end{document}