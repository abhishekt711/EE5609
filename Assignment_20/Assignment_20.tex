                                                                                                                                                                                                                                                                                                                                                                                                                                                                                                                                                                                                                                                                                \documentclass[journal,12pt,twocolumn]{IEEEtran}

\usepackage{setspace}
\usepackage{gensymb}

\singlespacing


\usepackage[cmex10]{amsmath}

\usepackage{amsthm}

\usepackage{mathrsfs}
\usepackage{txfonts}
\usepackage{stfloats}
\usepackage{bm}
\usepackage{cite}
\usepackage{cases}
\usepackage{subfig}

\usepackage{longtable}
\usepackage{multirow}

\usepackage{enumitem}
\usepackage{mathtools}
\usepackage{steinmetz}
\usepackage{tikz}
\usepackage{circuitikz}
\usepackage{verbatim}
\usepackage{tfrupee}
\usepackage[breaklinks=true]{hyperref}

\usepackage{tkz-euclide}

\usetikzlibrary{calc,math}
\usepackage{listings}
    \usepackage{color}                                            %%
    \usepackage{array}                                            %%
    \usepackage{longtable}                                        %%
    \usepackage{calc}                                             %%
    \usepackage{multirow}                                         %%
    \usepackage{hhline}                                           %%
    \usepackage{ifthen}                                           %%
    \usepackage{lscape}  
    \usepackage{blkarray}   
\usepackage{multicol}
\usepackage{chngcntr}
\usepackage{graphicx}
\DeclareMathOperator*{\Res}{Res}

\renewcommand\thesection{\arabic{section}}
\renewcommand\thesubsection{\thesection.\arabic{subsection}}
\renewcommand\thesubsubsection{\thesubsection.\arabic{subsubsection}}

\renewcommand\thesectiondis{\arabic{section}}
\renewcommand\thesubsectiondis{\thesectiondis.\arabic{subsection}}
\renewcommand\thesubsubsectiondis{\thesubsectiondis.\arabic{subsubsection}}


\hyphenation{op-tical net-works semi-conduc-tor}
\def\inputGnumericTable{}                                 %%

\lstset{
%language=C,
frame=single, 
breaklines=true,
columns=fullflexible
}
\begin{document}


\newtheorem{theorem}{Theorem}[section]
\newtheorem{problem}{Problem}
\newtheorem{proposition}{Proposition}[section]
\newtheorem{lemma}{Lemma}[section]
\newtheorem{corollary}[theorem]{Corollary}
\newtheorem{example}{Example}[section]
\newtheorem{definition}[problem]{Definition}

\newcommand{\BEQA}{\begin{eqnarray}}
\newcommand{\EEQA}{\end{eqnarray}}
\newcommand{\define}{\stackrel{\triangle}{=}}
\bibliographystyle{IEEEtran}
\providecommand{\mbf}{\mathbf}
\providecommand{\pr}[1]{\ensuremath{\Pr\left(#1\right)}}
\providecommand{\qfunc}[1]{\ensuremath{Q\left(#1\right)}}
\providecommand{\sbrak}[1]{\ensuremath{{}\left[#1\right]}}
\providecommand{\lsbrak}[1]{\ensuremath{{}\left[#1\right.}}
\providecommand{\rsbrak}[1]{\ensuremath{{}\left.#1\right]}}
\providecommand{\brak}[1]{\ensuremath{\left(#1\right)}}
\providecommand{\lbrak}[1]{\ensuremath{\left(#1\right.}}
\providecommand{\rbrak}[1]{\ensuremath{\left.#1\right)}}
\providecommand{\cbrak}[1]{\ensuremath{\left\{#1\right\}}}
\providecommand{\lcbrak}[1]{\ensuremath{\left\{#1\right.}}
\providecommand{\rcbrak}[1]{\ensuremath{\left.#1\right\}}}
\theoremstyle{remark}
\newtheorem{rem}{Remark}
\newcommand{\sgn}{\mathop{\mathrm{sgn}}}
\providecommand{\abs}[1]{\left\vert#1\right\vert}
\providecommand{\res}[1]{\Res\displaylimits_{#1}} 
\providecommand{\norm}[1]{\left\lVert#1\right\rVert}
%\providecommand{\norm}[1]{\lVert#1\rVert}
\providecommand{\mtx}[1]{\mathbf{#1}}
\providecommand{\mean}[1]{E\left[ #1 \right]}
\providecommand{\fourier}{\overset{\mathcal{F}}{ \rightleftharpoons}}
%\providecommand{\hilbert}{\overset{\mathcal{H}}{ \rightleftharpoons}}
\providecommand{\system}{\overset{\mathcal{H}}{ \longleftrightarrow}}
	%\newcommand{\solution}[2]{\textbf{Solution:}{#1}}
\newcommand{\solution}{\noindent \textbf{Solution: }}
\newcommand{\cosec}{\,\text{cosec}\,}
\providecommand{\dec}[2]{\ensuremath{\overset{#1}{\underset{#2}{\gtrless}}}}
\newcommand{\myvec}[1]{\ensuremath{\begin{pmatrix}#1\end{pmatrix}}}
\newcommand{\mydet}[1]{\ensuremath{\begin{vmatrix}#1\end{vmatrix}}}
\numberwithin{equation}{subsection}
\makeatletter
\@addtoreset{figure}{problem}
\makeatother
\let\StandardTheFigure\thefigure
\let\vec\mathbf
\renewcommand{\thefigure}{\theproblem}
\def\putbox#1#2#3{\makebox[0in][l]{\makebox[#1][l]{}\raisebox{\baselineskip}[0in][0in]{\raisebox{#2}[0in][0in]{#3}}}}
     \def\rightbox#1{\makebox[0in][r]{#1}}
     \def\centbox#1{\makebox[0in]{#1}}
     \def\topbox#1{\raisebox{-\baselineskip}[0in][0in]{#1}}
     \def\midbox#1{\raisebox{-0.5\baselineskip}[0in][0in]{#1}}
\vspace{3cm}
\title{EE5609 Assignment 20}
\author{Abhishek Thakur}
\maketitle
\bigskip
\renewcommand{\thefigure}{\theenumi}
\renewcommand{\thetable}{\theenumi}
\renewcommand{\labelenumi}{\alph{enumi})}
\begin{abstract}
This document solves problem based on Matrix Theory.
\end{abstract}
Download all solutions from 
\begin{lstlisting}
https://github.com/abhishekt711/EE5609/tree/master/Assignment_20
\end{lstlisting}
\section{Problem}
Consider a Markov Chain with state space $S = \cbrak{1,2, 3}$ and transition matrix
\begin{align} \nonumber
P = 
\begin{blockarray}{c@{\hspace{1pt}}rrr@{\hspace{3pt}}}
            & 1   & 2 & 3 \\
        \begin{block}{r@{\hspace{3pt}}@{\hspace{1pt}}
    (@{\hspace{1pt}}rrr@{\hspace{1pt}}@{\hspace{1pt}})}
        1 &  0 & \frac{1}{2} & \frac{1}{2}   \\ [2mm]
        2 & \frac{1}{2}  & 0 & \frac{1}{2}\\ [2mm]
        3 & \frac{1}{2}  &  \frac{1}{2} & 0  \\ [2mm]
%
        \end{block}
    \end{blockarray}
\end{align}
%
Let $\vec{\pi}$ be a stationary distribution of the Markov chain and $d(1)$ denote the
period of state 1.  Which of the following statements are correct?
\begin{enumerate}
\item $d(1) = 1$
\item $d(1) = 2$
\item $\pi_1 = \frac{1}{2}$
\item $\pi_1 = \frac{1}{3}$
\end{enumerate}
\section{solution}
\renewcommand{\thefigure}{1}
\begin{figure}
\begin{center}
\usetikzlibrary{automata, positioning}
    \begin{tikzpicture}[font=\sffamily]
        % Setup the style for the states
        \tikzset{node style/.style={state, 
                                    minimum width=1cm,
                                    line width=0.5mm,
                                    fill=gray!10!white}}

        % Draw the states
        \node[node style] at (0, 0)     (1)     {1};
        \node[node style] at (6, 0)     (2)     {2};
        \node[node style] at (3, -5.196) (3) 	{3};

        % Connect the states with arrows
        \draw[every loop,
              auto=right,
              line width=1mm,
              >=latex,
              draw=orange,
              fill=orange]
            (1)     edge[bend right=20]            node {$\frac{1}{2}$} (3)
            (1)     edge[bend right=20, auto=left] node {$\frac{1}{2}$} (2)
            (2)     edge[bend right=20]            node {$\frac{1}{2}$} (1)
            (2)     edge[bend right=20, auto=left] node {$\frac{1}{2}$} (3)
            (3) edge[bend right=20]            node {$\frac{1}{2}$} (1)
            (3) edge[bend right=20, auto=left] node {$\frac{1}{2}$} (2);
    \end{tikzpicture}  
    \caption{State transition diagram}
\end{center}
    \end{figure}    
    \begin{enumerate}
    \item{ The period of state 1 i.e, $d(1)$ is given as:
    \begin{align}
    d(1)=GCD\{n : p_{11}^n > 0\}\\
    d(1)=GCD\{2,3,4,\cdots\}\\
    \therefore d(1)=1 \label{eq1}
    \end{align}
    Thus statement a is correct}\\
    \item{As calucalted above in \ref{eq1}, $d(1)=1$\\
    Thus statement b is incorrect.}\\
    \item{For stationary distribution,
    \begin{align}
    \sum_{i=1}^{i=n} \pi_i = 1 \label{eq2}\\
    \pi   \vec{P} = \pi \label{eq3}      
    \end{align}
    Solving \ref{eq2} and \ref{eq3}, we get:
    \begin{align}
    \pi_1=\pi_2=\pi_3=\frac{1}{3} \label{eq4}
    \end{align}
    Thus option c is incorrect
    }\\
    \item{As, calculated in \ref{eq4}, $\pi_1=\frac{1}{3}$\\
    Thus option d is correct}
    \end{enumerate}
Hence, options a and d are correct.
\end{document}
