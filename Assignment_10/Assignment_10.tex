\documentclass[journal,12pt,twocolumn]{IEEEtran}

\usepackage{setspace}
\usepackage{gensymb}

\singlespacing


\usepackage[cmex10]{amsmath}

\usepackage{amsthm}

\usepackage{mathrsfs}
\usepackage{txfonts}
\usepackage{stfloats}
\usepackage{bm}
\usepackage{cite}
\usepackage{cases}
\usepackage{subfig}

\usepackage{longtable}
\usepackage{multirow}

\usepackage{enumitem}
\usepackage{mathtools}
\usepackage{steinmetz}
\usepackage{tikz}
\usepackage{circuitikz}
\usepackage{verbatim}
\usepackage{tfrupee}
\usepackage[breaklinks=true]{hyperref}

\usepackage{tkz-euclide}

\usetikzlibrary{calc,math}
\usepackage{listings}
    \usepackage{color}                                            %%
    \usepackage{array}                                            %%
    \usepackage{longtable}                                        %%
    \usepackage{calc}                                             %%
    \usepackage{multirow}                                         %%
    \usepackage{hhline}                                           %%
    \usepackage{ifthen}                                           %%
    \usepackage{lscape}     
\usepackage{multicol}
\usepackage{chngcntr}

\DeclareMathOperator*{\Res}{Res}

\renewcommand\thesection{\arabic{section}}
\renewcommand\thesubsection{\thesection.\arabic{subsection}}
\renewcommand\thesubsubsection{\thesubsection.\arabic{subsubsection}}

\renewcommand\thesectiondis{\arabic{section}}
\renewcommand\thesubsectiondis{\thesectiondis.\arabic{subsection}}
\renewcommand\thesubsubsectiondis{\thesubsectiondis.\arabic{subsubsection}}


\hyphenation{op-tical net-works semi-conduc-tor}
\def\inputGnumericTable{}                                 %%

\lstset{
%language=C,
frame=single, 
breaklines=true,
columns=fullflexible
}
\begin{document}


\newtheorem{theorem}{Theorem}[section]
\newtheorem{problem}{Problem}
\newtheorem{proposition}{Proposition}[section]
\newtheorem{lemma}{Lemma}[section]
\newtheorem{corollary}[theorem]{Corollary}
\newtheorem{example}{Example}[section]
\newtheorem{definition}[problem]{Definition}

\newcommand{\BEQA}{\begin{eqnarray}}
\newcommand{\EEQA}{\end{eqnarray}}
\newcommand{\define}{\stackrel{\triangle}{=}}
\bibliographystyle{IEEEtran}
\providecommand{\mbf}{\mathbf}
\providecommand{\pr}[1]{\ensuremath{\Pr\left(#1\right)}}
\providecommand{\qfunc}[1]{\ensuremath{Q\left(#1\right)}}
\providecommand{\sbrak}[1]{\ensuremath{{}\left[#1\right]}}
\providecommand{\lsbrak}[1]{\ensuremath{{}\left[#1\right.}}
\providecommand{\rsbrak}[1]{\ensuremath{{}\left.#1\right]}}
\providecommand{\brak}[1]{\ensuremath{\left(#1\right)}}
\providecommand{\lbrak}[1]{\ensuremath{\left(#1\right.}}
\providecommand{\rbrak}[1]{\ensuremath{\left.#1\right)}}
\providecommand{\cbrak}[1]{\ensuremath{\left\{#1\right\}}}
\providecommand{\lcbrak}[1]{\ensuremath{\left\{#1\right.}}
\providecommand{\rcbrak}[1]{\ensuremath{\left.#1\right\}}}
\theoremstyle{remark}
\newtheorem{rem}{Remark}
\newcommand{\sgn}{\mathop{\mathrm{sgn}}}
\providecommand{\abs}[1]{\left\vert#1\right\vert}
\providecommand{\res}[1]{\Res\displaylimits_{#1}} 
\providecommand{\norm}[1]{\left\lVert#1\right\rVert}
%\providecommand{\norm}[1]{\lVert#1\rVert}
\providecommand{\mtx}[1]{\mathbf{#1}}
\providecommand{\mean}[1]{E\left[ #1 \right]}
\providecommand{\fourier}{\overset{\mathcal{F}}{ \rightleftharpoons}}
%\providecommand{\hilbert}{\overset{\mathcal{H}}{ \rightleftharpoons}}
\providecommand{\system}{\overset{\mathcal{H}}{ \longleftrightarrow}}
	%\newcommand{\solution}[2]{\textbf{Solution:}{#1}}
\newcommand{\solution}{\noindent \textbf{Solution: }}
\newcommand{\cosec}{\,\text{cosec}\,}
\providecommand{\dec}[2]{\ensuremath{\overset{#1}{\underset{#2}{\gtrless}}}}
\newcommand{\myvec}[1]{\ensuremath{\begin{pmatrix}#1\end{pmatrix}}}
\newcommand{\mydet}[1]{\ensuremath{\begin{vmatrix}#1\end{vmatrix}}}
\numberwithin{equation}{subsection}
\makeatletter
\@addtoreset{figure}{problem}
\makeatother
\let\StandardTheFigure\thefigure
\let\vec\mathbf
\renewcommand{\thefigure}{\theproblem}
\def\putbox#1#2#3{\makebox[0in][l]{\makebox[#1][l]{}\raisebox{\baselineskip}[0in][0in]{\raisebox{#2}[0in][0in]{#3}}}}
     \def\rightbox#1{\makebox[0in][r]{#1}}
     \def\centbox#1{\makebox[0in]{#1}}
     \def\topbox#1{\raisebox{-\baselineskip}[0in][0in]{#1}}
     \def\midbox#1{\raisebox{-0.5\baselineskip}[0in][0in]{#1}}
\vspace{3cm}
\title{EE5609 Assignment 10}
\author{Abhishek Thakur}
\maketitle
\newpage
\bigskip
\renewcommand{\thefigure}{\theenumi}
\renewcommand{\thetable}{\theenumi}
\begin{abstract}
This document solves problem based on solution of vector space.
\end{abstract}
Download all solutions from 
\begin{lstlisting}
https://github.com/abhishekt711/EE5609/tree/master/Assignment_10
\end{lstlisting}
\section{Problem}
Consider the vectors in $\mathbb{R}^4$ defined by:\\
$\alpha_1=\myvec{-1\\0\\1\\2}$, $\alpha_2=\myvec{3\\4\\-2\\5}$ and $\alpha_3=\myvec{1\\4\\0\\9}$.\\
Find a system of homogeneous linear equations for which the space of solutions is exactly the subspace of $\mathbb{R}^4$ spanned by the given three vectors.\\
\section{Solution}
A system of linear equations is homogeneous if all of the constant terms are zero. It can be represented as,
\begin{align}
\vec{A}\vec{X}=0
\end{align}
Let $\vec{R}$  be  a echelon matrix which is reduced to  A.  Then  the  systems  $\vec{A}\vec{X}  =  0$  and  $\vec{R}\vec{X}  =  0$  have  the  same solutions.
Here, 
\begin{align}
\vec{A}=\myvec{-1&3&1\\0&4&4\\1&-2&0\\2&5&9}
\end{align}
By operating column operation on $\vec{A}$, we get:
\begin{align}
\myvec{-1&3&1\\0&4&4\\1&-2&0\\2&5&9}
 \xleftrightarrow{C_3=C_3-2C_1-C_2}\myvec{-1&3&0\\0&4&0\\1&-2&0\\2&5&0}\\
  \xleftrightarrow{C_1=-C_1}\myvec{1&3&0\\0&4&0\\-1&-2&0\\-2&5&0}\\
  \xleftrightarrow{C_2=C_2-3C_1}\myvec{1&0&0\\0&4&0\\-1&1&0\\-2&11&0}\\
  \xleftrightarrow{C_2=\frac{1}{4}C_2}\myvec{1&0&0\\0&1&0\\-1&\frac{1}{4}&0\\-2&\frac{11}{4}&0}\label{eq14}
\end{align}
The bais vector is non zero vector which are given from \ref{eq14},\\
\begin{align}
\rho_1=\myvec{1\\0\\-1\\-2}, \rho_2=\myvec{0\\1\\\frac{1}{4}\\\frac{11}{4}}
\end{align}
Vector $\vec{V}$ formed using the above basis vectors are given as,
\begin{align}
c_1\rho_1+c_2\rho_2=\myvec{c_1\\c_2\\-c_1+\frac{1}{4}c_2\\-2c_2+\frac{11}{4}}
\end{align}
If $\vec{X}$ is in $\vec{V}$, so it can be written in this form.\\
\begin{align}
\implies \myvec{1&0\\0&1\\-1&\frac{1}{4}\\-2&\frac{11}{4}}\myvec{c_1\\c_2}=\vec{X} \label{eq15} 
\end{align}
The augmented matrix using \ref{eq15} can be written as,
\begin{align}
\myvec{1&0&x_1\\0&1&x_2\\-1&\frac{1}{4}&x_3\\-2&\frac{11}{4}&x_4}
\end{align}
Converting the above augmented matrix into row reduced echelon form:
\begin{align}
\myvec{1&0&x_1\\0&1&x_2\\-1&\frac{1}{4}&x_3\\-2&\frac{11}{4}&x_4}\xleftrightarrow{R_3=R_3+R_1-\frac{1}{4}R_2, R_4=R_4+2R_1-\frac{11}{4}R_2} \nonumber\\
\myvec{1&0&x_1\\0&1&x_2\\0&0&x_1-\frac{1}{4}x_2+x_3\\0&0&2x_1-\frac{11}{4}x_2+x_4}\label{eq16}
\end{align}
From \ref{eq16}, the basis vector for the solution space is $\myvec{x_1\\x_2}$. The solution space is given as,
\begin{align}
\vec{X}=\myvec{x_1\\x_2\\x_1-\frac{1}{4}x_2\\2x_1-\frac{11}{4}x_2}
\end{align}
\end{document}